% Auteur : Steve Prud’Homme
% Cette oeuvre, création, site ou texte est sous licence Creative Commons Attribution - Pas d’Utilisation Commerciale - Partage dans les Mêmes Conditions 4.0 International. Pour accéder à une copie de cette licence, merci de vous rendre à l'adresse suivante 
% http://creativecommons.org/licenses/by-nc-sa/4.0/ ou envoyez un courrier à 
% Creative Commons, 444 Castro Street, Suite 900, Mountain View, California, 94041, USA.
\documentclass[aspectratio=169]{beamer}
%\usepackage{beamerthemesplit} % new 
\usepackage[french]{babel}
\usepackage[utf8]{inputenc}
\usepackage{tikz}
\usepackage[fixlanguage]{babelbib}
\selectbiblanguage{french}
% Natlib pour la bibliographie
\usepackage{natbib}
% Pour les hyperliens
\usepackage{url}
% pour les illustration
\usetikzlibrary{mindmap,shadows,shapes,backgrounds}
\usepackage[T1]{fontenc}
\setbeamertemplate{bibliography item}[text]
% Pour l'utilisation de colonnes multiples
\usepackage{multicol}

\begin{document}
	\title{L'évaluation en ligne : pour Alain Stockless} 
	\author{Steve Prud'Homme} 
	\institute{UQÀM - GTN-Québec - Commission scolaire de Laval} 
	\date{\today} 

	
	\frame{\titlepage} 
	
	\section{Contexte} 
		\begin{frame}[allowframebreaks]
			\frametitle{Le contexte de la formation à distance et en ligne}
			Le \citet{OLC2015a}, conclut dans une grande étude Étatsunienne qu'en 2015\footnote{Infographie : \url {http://www.onlinelearningsurvey.com/reports/2015SurveyInfo.pdf} } :
			\begin {itemize}
				\item Hausse d'inscriptions en formation à distance de 3,9 \% par rapport à 2014
				\item Plus du quart des étudiants (28 \%) prend des cours en formation à distance (un  total de 5 828 826 édutiants, une hausse annuelle de 217,275  pour une population de 321 773 631 [soit 1,8  \%])
				\item Le pourcentage de directeurs universitaires qui considèrent les résultats de l'apprentissage dans l'éducation en ligne comme étant égal ou supérieur à ceux de l'enseignement en face à face est désormais de 71,4 \%
				\item 29,1 \% des dirigeants universitaires rapportent que leur faculté reconnaît la « valeur et la légitimité de l'éducation en ligne ». Parmi les écoles avec les inscriptions les plus lointaines, 60,1 \% rapportent la reconnaissance par la faculté tandis que seules 11,6 \% des écoles avec des inscriptions locales le font.
			\end{itemize}
			\framebreak
			Selon la \citet{sofad2015a} en 2014 : 
			\begin {itemize}
				\item Le total des inscriptions à des cours à distance (formations générale et professionnelle) a augmenté de 3,6 \% et atteint un nouveau sommet à 56 608
				\item Le nombre d’élèves concernés par ces cours a augmenté quant à lui de 3 \% pour atteindre un nouveau sommet à 29 386 pour une population, selon \citet{is2015a}, de 8 263 600 (soit 0,36 \%)
				\item La croissance est particulièrement forte en formation professionnelle à distance avec des augmentations de 8 \% pour le nombre d’inscriptions et de 10 \% pour le nombre d’élèves, pour atteindre de nouveaux sommets dans un cas comme dans l’autre (10 209 et 3 374)
			\end{itemize}
		\end{frame}
		
			
	\subsection{Avantages et résultats} 
		\begin{frame}
			\frametitle{}
			\begin{columns}[t]
				\column{.5\textwidth}
					\begin{block}{Avantages \citep{Lamontagne2013}}
						\begin {itemize}
							\item Gestion facilitée
							\item Économie
							\item Standardisation
							\item Personnalisation
							\item Contrôle
							\item Rapidité
							\item Flexibilité
							\item Qualité
			\end{itemize}
					\end{block}
				\column{.5\textwidth}
					\begin{block}{Résultats}
					\citet {Stansbury2013C} et \citet{Lamontagne2013}, déclare en citant Scoot Smith, directeur TI à la Mooresville Graded School District (N.C), que :
						\begin {itemize}
							\item Les résultats aux tests d'état finaux connaissent des augmentations significatives (+20 \% sur 6 ans)
							\item Un taux de poursuite aux études supérieures de + 40 \%.
							\item Une augmentation de 16 \% dans la réussite globale
						\end{itemize}
			Selon \citet{Lamontagne2013}, ces améliorations sont dues à la systématisation des suivis et de l'attention accordée à chacun.
			
	
					\end{block}
		\end{columns}
			
		\end{frame}
	
	
	
	\section{Enjeux}
			
			\begin{frame}[allowframebreaks]
				  \frametitle{Enjeux}
				
				 
				 \begin{description}[Second Item]
					\item [Aisance TIC] \par Préparer le personnel enseignant, administratif ainsi que les élèves /étudiants)\citep{NorthCarolina2013}
					\item[Pédagogiques] Les enjeux pédagogiques sont multiples : 
					 	\begin{itemize}
					 		\item Facilite l'identification de ces objectifs, leur mise en relation et leur partage entre concepteurs          
							\item Soutient la diffusion aux étudiants et clarifient ainsi les attentes de l'évaluation
							\item Rend beaucoup plus simple la réalisation d'activités d'évaluation visant la démonstration de compétences diversifiées, comme la créativité ou la collaboration
							\item Vise des compétences de haut niveau (pas que la simple mémorisation)
							\item Favorise une évaluation plus formative et plus formatrice
							\item Facilite l'administration des tests diagnostiques et l'utilisation à plusieurs fins.					
							\item Facilite et accélèrent la transmission de la rétroaction
							\item Selon \citet{whitelock2006a}, cité par \citet{audet2011a}, cette rapidité de rétroaction est déterminante dans le développement de l'évaluation en ligne, car elle permet :
					 		\begin {itemize}
									\item L'automatisation
									\item La remise immédiate à l'apprenant
									\item La réutilisation
							\end{itemize}
							\item Rend l'évaluation plus captivante parce qu'elle permet d'y insérer des éléments multimédias ou des hyperliens.
						\item Permet la réalisation d'activité d'évaluation :
								\begin {itemize}
									\item Plus variées
									\item Plus authentiques
									\item Plus collaboratives
								\end{itemize}
						\item Presque que toutes les formes de démonstrations de compétences (exemple : les wikis)
						\item Il est facile de combiner des évaluations alternatives avec des tests formatifs ou sommatifs automatisés (diversité versus tâche de l'enseignant)						
						\item Selon \citet{audet2011a} : elle «[...] vient alors supporter la compilation et l'analyse des divers résultats obtenus »
						\item Facilite particulièrement les formes alternatives d'évaluation en permettant :
					 	
					 	\begin {itemize}
									\item L'anonymat
									\item La pondération complexe
									\item La compilation des résultats
									\item La discussion
									\item La négociation
									\item La conservation des traces
						\end{itemize}
						\item Utilise des critères détaillés facilités par la compilation efficace de résultats
						\item Élimine des contraintes liées à la diffusion des documents (diffusion des critères préalables et des résultats liés)
						\item  L'évaluation en ligne est souvent restreinte aux évaluations à portée faible\footnote{Habituellement formative et ses résultats demeurent locaux} et moyenne\footnote{Peuvent avoir des résultats locaux ou nationaux, mais ceux-ci ne sont pas déterminants dans la vie de la personne évaluée}.
						\item De plus en plus d'organisations l'envisagent dans des contextes de certification professionnelle, où l'impact d'un succès ou d'un échec est très important.	
						\end{itemize}								
					\item[Les enjeux technologiques] Selon la \citet{NorthCarolina2013}, citée par \citet{Lamontagne2013} il faut :
						\begin {itemize}
							\item Déternir un réseau développé.Selon \citet{Stansbury2013B}, une couverture complète du WiFi est nécessaire.			\item Diposer d'une large bande passante
							\item Posséder des équipements informatiques performants
						\end{itemize} 
					
					\item[Les enjeux économiques] Selon la \citet{NorthCarolina2013} :
					\begin {itemize}
						\item Réductions de coût au niveau national et local
						\item Temps réponse plus rapide pour les enseignants et les élèves / étudiants
					\end{itemize}
					\item[Les enjeux sociaux] Selon le \citet{NorthCarolina2013}, il y a une amélioration de l'accessibilité par l'usage d'accommodements pour l'élève ou l'étudiant (ex.: audio, vidéo, couleur arrière-plan alternative,etc.) 
				\end{description}
				
			\end{frame}
			
	\section{Pratiques} 
			
		\subsubsection{Tendances en évaluation en ligne} 
				\begin{frame}[allowframebreaks]
					\frametitle{Tendances en évaluation en ligne}				 	
				 	\begin {itemize}
						\item Utilisation d’un logiciel de contrôle à distance et d’un poste d’examen dédié disponible à distance
						\item Utilisation d'un environnement numérique d'apprentissage pour l’assignation des horaires d’examen			
						\item L’utilisation d’un questionnaire sur un environnement numérique d'apprentissage
						\item Utilisation Webcam et micro
						\item 2011-2012, en Caroline du Nord, 19 \% des examens EOC  (\textit{End-of-course}) sont administrés en ligne\citep{NorthCarolina2013}
						\item Plusieurs instances gouvernementales publient des guides de bonnes pratiques
						\item Aux États-Unis, 17 états sont membre du \textit{Smarter Balanced Assesment Consortium} qui a créé un système d'évaluation en ligne qui est arrimé au \textit{Common Core State Strandards} (CCSS). Ces 17 états ont des dispositisf centralisés d'évaluation en ligne pour les examens EOC.
						\item Au Royaumes-Unis, on rédige \textit{British Standards Institution code of practice for the use of information technology (IT) in the Delivery if assessments}\footnote{BS 7988:2002}. Ce code est devenu un standard international\footnote{ISO/IEC 23988:2007}
						\item Une grande place est accordée au portfolio numérique, car il permet à l'apprenant de démontrer ses connaissances et ses compétences tout en ayant une réflexion sur leur travail tout en étant créatif.
						\item Une certification créditée pour le personnel impliqué dans la prestation de service d'évaluation en ligne existe.
					\end{itemize}
\end{frame}
		
				
			

\section{Bibliographie}
\subsection{Bibliographie}
\frame[allowframebreaks]{\frametitle{Bibliographie}

\bibliographystyle{apalike}
\bibliography{bibliographie} %bibtex file name without .bib extension
}
\end{document}

